\documentclass[12pt,a4paper]{colorbook}
\usepackage{polyglossia}
\newfontfamily{\khmerfont}[Ligatures=TeX,AutoFakeBold=.5,AutoFakeSlant=.2]{Khmer OS}
\newfontfamily{\englishfont}{Times New Roman}
\setmainlanguage[numerals=khmer]{khmer}
\setotherlanguage{english}
\setsansfont[Ligatures=TeX,AutoFakeBold=0,AutoFakeSlant=.2]{Khmer OS Bokor}
\setmonofont[Ligatures=TeX,AutoFakeBold=.5,AutoFakeSlant=.2]{Khmer OS Freehand}
\begin{document}
	\frontmatter
	\tableofcontents
	\chapter{អារម្ភកថា}
	\chapter{សេចក្ដីថ្លែងអំណរគុណ}
	\mainmatter
	\part{ពិជគណិត}
	\chapter{ចំណងជើង}
	\section{ចំណងជើង}
	\subsection{ចំណងជើង}
	សរសេរខ្លឹមសារ
	\chapter{ចំណងជើង}
	\section{ចំណងជើង}
	\subsection{ចំណងជើង}
	សរសេរខ្លឹមសារ
	\chapter{ចំណងជើង}
	\section{ចំណងជើង}
	\subsection{ចំណងជើង}
	សរសេរខ្លឹមសារ
	\part{ធរណីមាត្រ}
	\chapter{ចំណងជើង}
	\section{ចំណងជើង}
	\subsection{ចំណងជើង}
	សរសេរខ្លឹមសារ
	\chapter{ចំណងជើង}
	\section{ចំណងជើង}
	\subsection{ចំណងជើង}
	សរសេរខ្លឹមសារ
	\chapter{ចំណងជើង}
	\section{ចំណងជើង}
	\subsection{ចំណងជើង}
	សរសេរខ្លឹមសារ
	\appendix
	\part{សេក្ដីបន្ថែម}
	\chapter{ចំណងជើង}
	\section{ចំណងជើង}
	\subsection{ចំណងជើង}
	សរសេរខ្លឹមសារ
	\chapter{ចំណងជើង}
	\section{ចំណងជើង}
	\subsection{ចំណងជើង}
	សរសេរខ្លឹមសារ
	\backmatter
	\begin{thebibliography}{9}
		\bibitem{wikibook} WiKiBook, \emph{https://en.wikibooks.org/wiki/LaTeX}
		\bibitem{tobias15} Tobias Oetiker, \emph{The Not So Short Introduction to \LaTeXe}, Version 5.05, July 18, 2015
		\bibitem{leslie94} Leslie Lamport, \emph{\LaTeX: A Document Preparation System}, 2nd Edition, Addison-Wesley Professional, 1994.
	\end{thebibliography}
\end{document}