\documentclass[12pt,a4paper]{book}
\usepackage{mystyle}
\everymath{\protect\displaystyle\protect\color{magenta}}
\begin{document}
	\frontmatter
	\tableofcontents
	\clearpage
	\chapter{អារម្ភកថា}
	សរសេរអារម្ភកថាទីនេះ
	\chapter{អំណរគុណ}
	សរសេរអំណរគុណទីនេះ
	\mainmatter
	\chapter{ចំនួនកុំផ្លិច}
	\section{ចំនួនកុំផ្លិចទម្រង់ពីជគណិត}
	\subsection{ប្រមាណវិធីបូកលើចំនួនកុំផ្លិច}
	\begin{dfn}
		លេខរៀង
		\begin{enumerate}
			\item ធាតុ
			\begin{enumerate}
				\item ធាតុ
				\begin{enumerate}
					\item ធាតុ
					\begin{enumerate}
						\item ធាតុ
						\item ធាតុ
					\end{enumerate}
					\item ធាតុ
				\end{enumerate}
				\item ធាតុ
			\end{enumerate}
			\item ធាតុ
		\end{enumerate}
	\end{dfn}
\section*{លំហាត់}
\begin{enumerate}
	\item ចូរសរសេរនិយមន័យដេរីវេនៃអនុគមន៍ $ f(x) $ ត្រង់ចំណុច $ a $។
	\item គណនាដេរីវេទី $ n $ នៃអនុគមន៍ខាងក្រោម៖
	\begin{tasks}[counter-format=tsk[k].](4)
		\task $ f(x)=e^x $
		\task $ f(x)=x e^x $
		\task $ f(x)=x^2 e^x $
		\task $ f(x)=x^3 e^x $
		\task $ f(x)=\cos x $
		\task $ f(x)=\sin x $
		\task $ f(x)=x\cos x $
		\task $ f(x)=x\sin x $
	\end{tasks}
\end{enumerate}
\chapter{ចំណងជើងជំពូក}
\section{ចំណងជើងផ្នែក}
\subsection{ចំណងជើងផ្នែករង}
	\begin{thm}
		ចំណុច
		\begin{itemize}
			\item ធាតុ
			\begin{itemize}
				\item ធាតុ
				\begin{itemize}
					\item ធាតុ
					\begin{itemize}
						\item ធាតុ
						\item ធាតុ
					\end{itemize}
					\item ធាតុ
				\end{itemize}
				\item ធាតុ
			\end{itemize}
			\item ធាតុ
		\end{itemize}
	\end{thm}
	\begin{proof}
		សរសេរទីនេះ
	\end{proof}
\appendix
\chapter{ចំណងជើងជំពូក}
\section{ចំណងជើងផ្នែក}
\subsection{ចំណងជើងផ្នែករង}
\backmatter
\begin{thebibliography}{3}
	\bibitem{wikibook} WiKiBook, \emph{https://en.wikibooks.org/wiki/LaTeX}
	\bibitem{tobias15} Tobias Oetiker, \emph{The Not So Short Introduction to \LaTeXe}, Version 5.05, July 18, 2015
	\bibitem{leslie94} Leslie Lamport, \emph{\LaTeX: A Document Preparation System}, 2nd Edition, Addison-Wesley Professional, 1994.
\end{thebibliography}
\end{document}