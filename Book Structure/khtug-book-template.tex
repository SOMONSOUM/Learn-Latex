\documentclass[a4paper,12pt]{book}
%
\usepackage{geometry}
\usepackage{polyglossia}
\usepackage{float}
\usepackage{graphicx}
\usepackage{subcaption}
\usepackage{fancyhdr}
\usepackage{titlesec}
\usepackage{multicol}
\usepackage{enumitem}
\usepackage{hyperref}
\usepackage{amsmath}
\usepackage{amssymb}
\usepackage{amsthm}
\usepackage{mathpazo}
\usepackage{emptypage}
\usepackage{tcolorbox}
\usepackage{tikz}
\usepackage{url}
%
\geometry{%
	top=1in,%
	bottom=1in,%
	left=.7in,%
	right=.7in}
%
\newfontfamily{\khmerfont}{Khmer OS Content}[%
	Script=Khmer,%
	Ligatures=TeX,%
	BoldFont={Khmer OS Bokor},%
	ItalicFont=*,%
	ItalicFeatures={FakeSlant=.15}]
\setmainlanguage[numerals=khmer]{khmer}
\setsansfont{Khmer OS}[%
	Script=Khmer,%
	Ligatures=TeX,%
	BoldFont={Khmer OS Bokor},%
	ItalicFont=*,%
	ItalicFeatures={FakeSlant=.15}]
\setmonofont{Khmer OS}[%
	Script=Khmer,%
	Ligatures=TeX,%
	BoldFont={Khmer OS Bokor},%
	ItalicFont=*,%
	ItalicFeatures={FakeSlant=.15}]
%
\theoremstyle{theorem}
\newtheorem{theorem}{ទ្រឹស្ដីបទ}[chapter]
\theoremstyle{definition}
\newtheorem{definition}{និយមន័យ}[chapter]
\theoremstyle{remark}
\newtheorem{remark}{សម្គាល់}[chapter]
%
\fancypagestyle{plain}{%
	\renewcommand{\headrulewidth}{1pt}%
	\renewcommand{\footrulewidth}{1pt}}
\pagestyle{plain}
%
\hypersetup{%
	hidelinks,%
	colorlinks,%
	linkcolor=blue,%
	citecolor=blue,%
	urlcolor=blue}
%
\begin{document}
	%
	\frontmatter
	\tableofcontents
	%
	\mainmatter
	%
	\chapter{ចំណងជើងជំពូក}
	\section{ចំណងជើង}
	\begin{definition}
		សរសេរខ្លឹមសារនៅទីនេះ សរសេរខ្លឹមសារនៅទីនេះ សរសេរខ្លឹមសារនៅទីនេះ សរសេរខ្លឹមសារនៅទីនេះ សរសេរខ្លឹមសារនៅទីនេះ សរសេរខ្លឹមសារនៅទីនេះ សរសេរខ្លឹមសារនៅទីនេះ សរសេរខ្លឹមសារនៅទីនេះ សរសេរខ្លឹមសារនៅទីនេះ សរសេរខ្លឹមសារនៅទីនេះ សរសេរខ្លឹមសារនៅទីនេះ សរសេរខ្លឹមសារនៅទីនេះ សរសេរខ្លឹមសារនៅទីនេះ សរសេរខ្លឹមសារនៅទីនេះ សរសេរខ្លឹមសារនៅទីនេះ 
	\end{definition}
	\section{ចំណងជើង}
	\begin{theorem}
		សរសេរខ្លឹមសារនៅទីនេះ សរសេរខ្លឹមសារនៅទីនេះ សរសេរខ្លឹមសារនៅទីនេះ សរសេរខ្លឹមសារនៅទីនេះ សរសេរខ្លឹមសារនៅទីនេះ សរសេរខ្លឹមសារនៅទីនេះ សរសេរខ្លឹមសារនៅទីនេះ សរសេរខ្លឹមសារនៅទីនេះ សរសេរខ្លឹមសារនៅទីនេះ សរសេរខ្លឹមសារនៅទីនេះ សរសេរខ្លឹមសារនៅទីនេះ សរសេរខ្លឹមសារនៅទីនេះ សរសេរខ្លឹមសារនៅទីនេះ សរសេរខ្លឹមសារនៅទីនេះ សរសេរខ្លឹមសារនៅទីនេះ 
	\end{theorem}
	%
	\chapter{ចំណងជើងជំពូក}
	\section{ចំណងជើង}
	\begin{proof}
		សរសេរខ្លឹមសារនៅទីនេះ សរសេរខ្លឹមសារនៅទីនេះ សរសេរខ្លឹមសារនៅទីនេះ សរសេរខ្លឹមសារនៅទីនេះ សរសេរខ្លឹមសារនៅទីនេះ សរសេរខ្លឹមសារនៅទីនេះ សរសេរខ្លឹមសារនៅទីនេះ សរសេរខ្លឹមសារនៅទីនេះ សរសេរខ្លឹមសារនៅទីនេះ សរសេរខ្លឹមសារនៅទីនេះ សរសេរខ្លឹមសារនៅទីនេះ សរសេរខ្លឹមសារនៅទីនេះ សរសេរខ្លឹមសារនៅទីនេះ សរសេរខ្លឹមសារនៅទីនេះ សរសេរខ្លឹមសារនៅទីនេះ 
	\end{proof}
	\section{ចំណងជើង}
	\begin{remark}
		សរសេរខ្លឹមសារនៅទីនេះ សរសេរខ្លឹមសារនៅទីនេះ សរសេរខ្លឹមសារនៅទីនេះ សរសេរខ្លឹមសារនៅទីនេះ សរសេរខ្លឹមសារនៅទីនេះ សរសេរខ្លឹមសារនៅទីនេះ សរសេរខ្លឹមសារនៅទីនេះ សរសេរខ្លឹមសារនៅទីនេះ សរសេរខ្លឹមសារនៅទីនេះ សរសេរខ្លឹមសារនៅទីនេះ សរសេរខ្លឹមសារនៅទីនេះ សរសេរខ្លឹមសារនៅទីនេះ សរសេរខ្លឹមសារនៅទីនេះ សរសេរខ្លឹមសារនៅទីនេះ សរសេរខ្លឹមសារនៅទីនេះ 
	\end{remark}
	%
	\appendix
	\chapter{ចំណងជើងជំពូក}
	\section{ចំណងជើង}
	សរសេរខ្លឹមសារនៅទីនេះ សរសេរខ្លឹមសារនៅទីនេះ សរសេរខ្លឹមសារនៅទីនេះ សរសេរខ្លឹមសារនៅទីនេះ សរសេរខ្លឹមសារនៅទីនេះ សរសេរខ្លឹមសារនៅទីនេះ សរសេរខ្លឹមសារនៅទីនេះ សរសេរខ្លឹមសារនៅទីនេះ សរសេរខ្លឹមសារនៅទីនេះ សរសេរខ្លឹមសារនៅទីនេះ សរសេរខ្លឹមសារនៅទីនេះ សរសេរខ្លឹមសារនៅទីនេះ សរសេរខ្លឹមសារនៅទីនេះ សរសេរខ្លឹមសារនៅទីនេះ សរសេរខ្លឹមសារនៅទីនេះ 
	\section{ចំណងជើង}
	សរសេរខ្លឹមសារនៅទីនេះ សរសេរខ្លឹមសារនៅទីនេះ សរសេរខ្លឹមសារនៅទីនេះ សរសេរខ្លឹមសារនៅទីនេះ សរសេរខ្លឹមសារនៅទីនេះ សរសេរខ្លឹមសារនៅទីនេះ សរសេរខ្លឹមសារនៅទីនេះ សរសេរខ្លឹមសារនៅទីនេះ សរសេរខ្លឹមសារនៅទីនេះ សរសេរខ្លឹមសារនៅទីនេះ សរសេរខ្លឹមសារនៅទីនេះ សរសេរខ្លឹមសារនៅទីនេះ សរសេរខ្លឹមសារនៅទីនេះ សរសេរខ្លឹមសារនៅទីនេះ សរសេរខ្លឹមសារនៅទីនេះ 
	\backmatter
	\begin{thebibliography}{99}
		\bibitem{lamport94}
		Leslie Lamport,
		\emph{\LaTeX: a document preparation system},
		Addison Wesley, Massachusetts,
		2nd edition,
		1994.
	\end{thebibliography}
	%
\end{document}