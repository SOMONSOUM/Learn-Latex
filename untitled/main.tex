%% ----------------------------------------------------------------------
%% Esta es una plantilla para la clase de documento exam, en español
%% ----------------------------------------------------------------------
%% 2017 por Fausto M. Lagos S. <piratax007@protonmail.ch>
%% 
%% Este trabajo puede ser distribuido o modificado bajo los
%% términos y condiciones de la LaTeX Project Public License (LPPL) v1.3C, 
%% o cualquier versión posterior. La última versión de esta licencia
%% puede verse en:
%% http://www.latex-project.org/lppl.txt
%% 
%% Usted es libre de usarlo, modificarlo o distribuirlo siempre que se
%% respeten los términos de la licencia y se reconozca al autor original
%% ----------------------------------------------------------------------

%Preámbulo
\documentclass[12pt, legalpaper]{exam}
\usepackage[utf8]{inputenc}
\usepackage[spanish]{babel}
\usepackage[margin=.75in]{geometry}
\usepackage{amsmath,amssymb}
\usepackage{multicol}
\usepackage{graphicx}
\usepackage{tikz}

% Configuración del Encabezado
\newcommand{\class}{Tema}
\newcommand{\term}{Periodo, Semestre o Curso}
\newcommand{\examnum}{Número del Parcial}
\newcommand{\examdate}{Fecha:}
\newcommand{\timelimit}{Tiempo disponible}
\pagestyle{head}
\firstpageheader{Institución}{}{Departamento}
\runningheader{\class}{\examnum\ - Page \thepage\ of \numpages}{\examdate}
\runningheadrule

% Configuración de la tabla de calificación
\pointpoints{punto}{puntos}
\hpword{Puntos:}
\vpword{Puntos}
\vtword{Total:}
\htword{Total}
\vsword{Resultado}
\hsword{Resultado:}
\vqword{Problema}
\hqword{Pregunta:}

\begin{document}
% Definición del Encabezado
\noindent
\begin{tabular*}{\textwidth}{l @{\extracolsep{\fill}} r @{\extracolsep{6pt}} l}
\textbf{\class} & \textbf{Nombre:} & \makebox[2.5in]{\hrulefill}\\
\textbf{\term} &&\\
\textbf{\examnum} & \textbf{Curso:} & \makebox[2.5in]{\hrulefill}\\
\textbf{\examdate} &&\\
\textbf{Tiempo: \timelimit} & Profesor: & \makebox[2.5in]{\emph{Dr. John Doe}}
\end{tabular*}\\
\rule[2ex]{\textwidth}{2pt}

% Bloque de instrucciones
\noindent
Este examen contiene \numquestions \;planteamientos que corresponde a \numpoints \;puntos de la valoración final. Tenga presente que no esta autorizada la comunicación con sus compañeros, ni el uso de ayudas computacionales (calculadora, celular, etc) y que resolver el pliego a l\'apiz implica renunciar a cualquier reclamación después de entregados los resultados.

% Tabla de calificaciones
\begin{center}
Tabla de calificación de uso exclusivo para el profesor. \\
\addpoints
% Puede presentar la tabla de calificación en orientación vertical [v] u horizontal [h]
\gradetable[h][questions]
\end{center}

% Preguntas de ejemplo.
\begin{questions}
\addpoints
\question[25]  En la funci\'on dada se garantiza que hay tres puntos de inflexi\'on ubicados en las raices o ceros y el $y_i$. Determine: Dominio, Rango, Tipo de funci\'on (inyectiva, sobreyectiva o biyectiva), paridad e intervalos en los que es creciente o decreciente y construya una aproximaci\'on gr\'afica del lugar geom\'etrico de la funci\'on.
\[
	f(x) = x^4 - 4x^2 + 4
\]
\emph{Sugerencia:} Para determinar las raices o ceros resuelva la ecuaci\'on $x^4 - 4x^2 + 4 = 0$

\emph{Observaci\'on:} La determinaci\'on de la paridad (en particular) debe estar apoyada en el procedimiento algebraico que la sustenta. Dado que conoce con exactitud los puntos de inflexi\'on, los intervalos en que la curva es creciente o decreciente deben darse exactamente definidos.
\addpoints

\addpoints
\question[25] Resuelva anal\'iticamente
	\[
		\lim _{x \to 1} \frac{x^3 - 1}{x^2 - 1}
	\]

\addpoints
\question[25] Determine, si los hay, los n\'umeros en los que la funci\'on dada es discontinua.
	\[
		f(x) = (x^2 - 9x + 18)^{-1}
	\]

\addpoints
\question[25] Encuentre $D_xy$
\[
	y = \frac{(x + 1)^2}{3x - 4}
\]

\end{questions}

\noindent
\rule[2ex]{\textwidth}{2pt}
\begin{center}
	\textbf{Soluci\'on}
\end{center}

\end{document}
