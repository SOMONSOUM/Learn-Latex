\documentclass[12pt,a4paper]{article}
\usepackage{polyglossia}
\newfontfamily{\khmerfont}[%
Script=Khmer,%
Ligatures=TeX,%
BoldFont=*,%
ItalicFont=*,%
BoldItalicFont=*%
]{Khmer OS}
\setsansfont[%
Script=Khmer,%
Ligatures=TeX,%
BoldFont=*,%
ItalicFont=*,%
BoldItalicFont=*%
]{Khmer OS Bokor}
\setmonofont[%
Script=Khmer,%
Ligatures=TeX,%
BoldFont=*,%
ItalicFont=*,%
BoldItalicFont=*%
]{Khmer OS Freehand}
\newfontfamily{\bbfamily}[%
Script=Khmer,%
Ligatures=TeX,%
BoldFont=*,%
ItalicFont=*,%
BoldItalicFont=*%
]{Khmer OS Battambang}
\newfontfamily{\ctfamily}[%
Script=Khmer,%
Ligatures=TeX,%
BoldFont=*,%
ItalicFont=*,%
BoldItalicFont=*%
]{Khmer OS Content}
\newfontfamily{\fhfamily}[%
Script=Khmer,%
Ligatures=TeX,%
BoldFont=*,%
ItalicFont=*,%
BoldItalicFont=*%
]{Khmer OS Fasthand}
\newfontfamily{\mcfamily}[%
Script=Khmer,%
Ligatures=TeX,%
BoldFont=*,%
ItalicFont=*,%
BoldItalicFont=*%
]{Khmer OS Metal Chrieng}
\newfontfamily{\mlfamily}[%
Script=Khmer,%
Ligatures=TeX,%
BoldFont=*,%
ItalicFont=*,%
BoldItalicFont=*%
]{Khmer OS Muol}
\newfontfamily{\mpfamily}[%
Script=Khmer,%
Ligatures=TeX,%
BoldFont=*,%
ItalicFont=*,%
BoldItalicFont=*%
]{Khmer OS Muol Pali}
\newfontfamily{\srfamily}[%
Script=Khmer,%
Ligatures=TeX,%
BoldFont=*,%
ItalicFont=*,%
BoldItalicFont=*%
]{Khmer OS Siemreap}
\newcommand{\textbb}[1]{\begingroup\bbfamily#1\endgroup}
\newcommand{\textct}[1]{\begingroup\ctfamily#1\endgroup}
\newcommand{\textfh}[1]{\begingroup\fhfamily#1\endgroup}
\newcommand{\textmc}[1]{\begingroup\mcfamily#1\endgroup}
\newcommand{\textml}[1]{\begingroup\mlfamily#1\endgroup}
\newcommand{\textmp}[1]{\begingroup\mpfamily#1\endgroup}
\newcommand{\textsr}[1]{\begingroup\srfamily#1\endgroup}
\newfontfamily{\englishfont}{Times New Roman}
\setmainlanguage[numerals=khmer]{khmer}
\setotherlanguage{english}
\let\texten\textenglish
\renewcommand{\arraystretch}{1.7}
\usepackage{tabulary}
\begin{document}
	\begin{table}[h]
		\centering
		\begin{tabulary}{\textwidth}{|L|L|}
			\hline
			ឈ្មោះពុម្ភអក្សរ & អត្ថបទគំរូ\\
			\hline
			\texten{Khmer OS} & តើមានអ្វីប្លែក?\\
			\texten{Khmer OS Bokor} & \textsf{តើមានអ្វីប្លែក?}\\
			\texten{Khmer OS Freehand} & \texttt{តើមានអ្វីប្លែក?}\\
			\hline
			\texten{Khmer OS Battambang} & \textbb{តើមានអ្វីប្លែក?}\\
			\texten{Khmer OS Content} & \textct{តើមានអ្វីប្លែក?}\\
			\texten{Khmer OS Fasthand} & \textfh{តើមានអ្វីប្លែក?}\\
			\texten{Khmer OS Metal Chrieng} & \textmc{តើមានអ្វីប្លែក?}\\
			\texten{Khmer OS Muol} & \textml{តើមានអ្វីប្លែក?}\\
			\texten{Khmer OS Muol Pali} & \textmp{តើមានអ្វីប្លែក?}\\
			\texten{Khmer OS Siemreap} & \textsr{តើមានអ្វីប្លែក?}\\
			\hline
		\end{tabulary}
		\caption{ការប្រើប្រាស់ពុម្ភអក្សរផ្សេងៗ}
	\end{table}
\end{document}