\documentclass{beamer}
\hypersetup{pdfpagemode=FullScreen}
\usetheme{Madrid}%You can choose any theme you like the most
\usepackage{polyglossia}%You can translate from English to any language
\usepackage{multicol}
\newfontfamily\khmerfont[Script=Khmer]{Khmer OS Content}
\newfontfamily\englishfont{Latin Modern Roman}
\setdefaultlanguage[numerals=khmer]{khmer}
\setsansfont[Script=Khmer]{Khmer OS Content}
\setmonofont[Script=Khmer]{Khmer OS Content}
\XeTeXlinebreaklocale "KHM"
\usepackage{ucharclasses}
\setTransitionsForLatin{\begingroup\englishfont}{\endgroup}
%\title{\LaTeXe\ ភាសាខ្មែរ}
\title{លីមីត}
\subtitle[\LaTeX]{ការសរសេរឯកសារភាសាខ្មែរជាមួយកម្មវិធី \LaTeX}
\author[ស៊ុំ សំអុន]{ការប្រើប្រាស់ \LaTeX{} ជាភាសាខ្មែរ}
\institute{វិទ្យាស្ថានបច្ចេកវិទ្យាកម្ពុជា​}

\begin{document}
	\begin{frame}{សូមស្វាគមន៍!}
		\titlepage
	\end{frame}
	\begin{frame}{មាតិការ}
		\tableofcontents
	\end{frame}
	\section{សេចក្តីចាប់ផ្ដើម}
	\begin{frame}{លីមីតនៃអនុគមន៍ត្រីកោណមាត្រ}
		\begin{block}{គណនាលីមីតនៃអនុគមន៍ខាងក្រោម៖}
		\begin{multicols}{2}
			\begin{enumerate}[a)]
				%\item $ \lim\limits_{x\to 0} \dfrac{\sin10x}{\sin5x} $
				\item $ \lim\limits_{x\to 0} \dfrac{\sin4x}{\sin2x} $
				\item $ \lim\limits_{x\to 0} \dfrac{x\sin5x}{\sin^2 4x} $
				\item $ \lim\limits_{x\to 0} \dfrac{\sqrt{7}-\sqrt{7\cos5x}}{1-\cos4x} $
				\item $ \lim\limits_{x\to 0} \dfrac{\sqrt{\sin2x}-\sqrt{\sin2x \cos4x}}{2x} $
				\item $ \lim\limits_{x\to 0} \dfrac{1-\cos3x}{1-\cos2x} $
				\item $ \lim\limits_{x\to 0} \dfrac{1-\cos^2 4x}{x\sin4x} $
			\end{enumerate}
		\end{multicols}
		\end{block}
	%
		\begin{block}{រូបមន្តគោលដែលត្រូវប្រើ}
			\begin{itemize}
				\item $ \lim\limits_{u\to 0} \dfrac{\sin u}{u}=1$ ឬ $  \lim\limits_{u\to 0} \dfrac{\sin^n u}{u^n}=1 $ ដែល $ n\geq1 $
				\item $ 1-\cos^2 ax=\sin^2 ax $
				\item $ 1-\cos ax=2\sin^2 \dfrac{a}{2}x $
				\item $ \left(a-b\right)^2 = a^2 -2ab +b^2$
			\end{itemize}
		\end{block}
\end{frame}
%
	\section{ការដំឡើងកម្មវិធី \LaTeX{}}
	\begin{frame}{របៀបដំឡើង \LaTeX{}}
		\begin{definition}[កម្មវិធី \LaTeX{}]
			សរសេរខ្លឹមសារទីនេះ
		\end{definition}
		%
		\begin{example}[ការណែនាំ]
		សរសេរខ្លឹមសារទីនេះ
		\end{example}
	\end{frame}
		\subsection{ចំពោះម៉ាស៊ីនប្រតិបត្តការណ៍ Windows}
		\begin{frame}{ម៉ាស៊ីនប្រតិបត្តការណ៍ Windows}
			\begin{columns}
				%
				\begin{column}{0.47\textwidth}
					\begin{theorem}
						សរសេរខ្លឹមសារទីនេះ
					\end{theorem}
				\end{column}
				%
				\begin{column}{0.47\textwidth}
					\begin{proof}
						សរសេរខ្លឹមសារទីនេះ
					\end{proof}
				\end{column}
			\end{columns}
		\end{frame}
	\subsection{ចំពោះម៉ាស៊ីនប្រតិបត្តការណ៍ Mac}
		\begin{frame}{ម៉ាស៊ីនប្រតិបត្តការណ៍ Mac}
			លោកអ្នកអាចចូរទៅកាន់ Website​.....
		\end{frame}
	\subsection{ចំពោះម៉ាស៊ីនប្រតិបត្តការណ៍ Unix}
		\begin{frame}{ម៉ាស៊ីនប្រតិបត្តការណ៍ Unix}
			\begin{enumerate}
				\item ចំណុច
			\begin{enumerate}[a)]
				\item ចំណុច
				\item ចំណុច
				\item ចំណុច
			\end{enumerate}
			\item ចំណុច
			\begin{enumerate}[a)]
				\item ចំណុច
				\item ចំណុច
				\item ចំណុច
			\end{enumerate}
			\item ចំណុច
			\begin{enumerate}[a)]
				\item ចំណុច
				\item ចំណុច
				\item ចំណុច
			\end{enumerate}
		\end{enumerate}
	\end{frame}
	\section{ឯកសារយោង}
	\begin{frame}{ប្រភពនៃឯកសារ}
		\begin{thebibliography}{4}
			\bibitem{lamport94}
			Leslie Lamport,
			\emph{\LaTeX: a document preparation system}.
			Addison Wesley, Massachusetts,
			2nd edition,
			1994.
				\bibitem{lamport94}
			Leslie Lamport,
			\emph{\LaTeX: a document preparation system}.
			Addison Wesley, Massachusetts,
			2nd edition,
			1994.
		\end{thebibliography}
	\end{frame}
\end{document}
