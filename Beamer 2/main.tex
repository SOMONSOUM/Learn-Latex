\documentclass{beamer}
\hypersetup{pdfpagemode=FullScreen}
\usetheme{Madrid}%You can choose any theme you like the most
\usepackage{polyglossia}%You can translate from English to any language
\newfontfamily\khmerfont[Script=Khmer]{Khmer OS Content}
\newfontfamily\englishfont{Latin Modern Roman}
\setdefaultlanguage[numerals=khmer]{khmer}
\setsansfont[Script=Khmer]{Khmer OS Content}
\setmonofont[Script=Khmer]{Khmer OS Content}
\XeTeXlinebreaklocale "KHM"
\usepackage{ucharclasses}
\setTransitionsForLatin{\begingroup\englishfont}{\endgroup}
\title{\LaTeXe\ ភាសាខ្មែរ}
\author[តិចណូ]{ការប្រើប្រាស់ \LaTeX{} ជាភាសាខ្មែរ}
\institute{វិទ្យាស្ថានបច្ចេកវិទ្យាកម្ពុជា}
\begin{document}
	\begin{frame}{សូមស្វាគមន៍!}
		\titlepage
	\end{frame}
	\begin{frame}{មាតិការ}
		\tableofcontents
	\end{frame}
	\section{សេចក្តីចាប់ផ្ដើម}
	\begin{frame}{ការណែនាំឲ្យស្គាល់អំពី \LaTeX{}}
		\begin{block}{តើអ្វីទៅជាកម្មវិធី \LaTeX{}?}
		សរសេរខ្លឹមសារទីនេះ
		\end{block}
	%
		\begin{block}{}
		សរសេរខ្លឹមសារទីនេះ
		\end{block}
\end{frame}
%
	\section{ការដំឡើងកម្មវិធី \LaTeX{}}
	\begin{frame}{របៀបដំឡើង \LaTeX{}}
		\begin{definition}[កម្មវិធី \LaTeX{}]
			សរសេរខ្លឹមសារទីនេះ
		\end{definition}
		%
		\begin{example}[ការណែនាំ]
		សរសេរខ្លឹមសារទីនេះ
		\end{example}
	\end{frame}
		\subsection{ចំពោះម៉ាស៊ីនប្រតិបត្តការណ៍ Windows}
		\begin{frame}{ម៉ាស៊ីនប្រតិបត្តការណ៍ Windows}
			\begin{columns}
				%
				\begin{column}{0.47\textwidth}
					\begin{theorem}
						សរសេរខ្លឹមសារទីនេះ
					\end{theorem}
				\end{column}
				%
				\begin{column}{0.47\textwidth}
					\begin{proof}
						សរសេរខ្លឹមសារទីនេះ
					\end{proof}
				\end{column}
			\end{columns}
		\end{frame}
	\subsection{ចំពោះម៉ាស៊ីនប្រតិបត្តការណ៍ Mac}
		\begin{frame}{ម៉ាស៊ីនប្រតិបត្តការណ៍ Mac}
			លោកអ្នកអាចចូរទៅកាន់ Website​.....
		\end{frame}
	\subsection{ចំពោះម៉ាស៊ីនប្រតិបត្តការណ៍ Unix}
		\begin{frame}{ម៉ាស៊ីនប្រតិបត្តការណ៍ Unix}
			\begin{enumerate}
				\item ចំណុច
			\begin{enumerate}[a)]
				\item ចំណុច
				\item ចំណុច
				\item ចំណុច
			\end{enumerate}
			\item ចំណុច
			\begin{enumerate}[a)]
				\item ចំណុច
				\item ចំណុច
				\item ចំណុច
			\end{enumerate}
			\item ចំណុច
			\begin{enumerate}[a)]
				\item ចំណុច
				\item ចំណុច
				\item ចំណុច
			\end{enumerate}
		\end{enumerate}
	\end{frame}
	\section{ឯកសារយោង}
	\begin{frame}{ប្រភពនៃឯកសារ}
		\begin{thebibliography}{4}
			\bibitem{lamport94}
			Leslie Lamport,
			\emph{\LaTeX: a document preparation system}.
			Addison Wesley, Massachusetts,
			2nd edition,
			1994.
				\bibitem{lamport94}
			Leslie Lamport,
			\emph{\LaTeX: a document preparation system}.
			Addison Wesley, Massachusetts,
			2nd edition,
			1994.
		\end{thebibliography}
	\end{frame}
\end{document}
