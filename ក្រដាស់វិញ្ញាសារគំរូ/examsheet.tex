\documentclass[a4paper,12pt]{article}
\makeatletter
\usepackage{multicol}% multi columns
\def\@school{}
\def\@subject{}
\def\@teacher{}
\def\school#1{\def\@school{#1}}
\def\subject#1{\def\@subject{#1}}
\def\teacher#1{\def\@teacher{#1}}
\def\label@school{ប្រឡង}
\def\label@subject{មុខវិជ្ជា}
\def\label@teacher{រៀបរៀងដោយ}
\def\label@point{ពិន្ទុ}
\school{តេស្តសមត្ថភាព}
\teacher{ស៊ុំ សំអុន}
\subject{គណិតវិទ្យា}
\usepackage{geometry}
\geometry{top=.8in,bottom=.8in,left=.4in,right=.4in}
\usepackage{polyglossia}
\newfontfamily{\khmerfont}[BoldFont={Khmer OS Bokor}]{Khmer OS Content}
\setmainlanguage[numerals=khmer]{khmer}
\usepackage{xcolor}
\usepackage{fancyhdr}
\fancyhf{}
\fancyhead[L]{\color{blue}\bfseries\label@school\space\@school}
\fancyhead[R]{\color{blue}\bfseries\label@subject\space\@subject}
\fancyfoot[C]{\color{blue}\bfseries\label@teacher\space\@teacher}
\renewcommand{\headrulewidth}{1pt}
\renewcommand{\footrulewidth}{1pt}
\pagestyle{fancy}
\usepackage[inline]{enumitem}
\setlist[itemize]{leftmargin=*,itemsep=\z@,labelsep=2\p@}
\setlist[itemize,1]{label=\protect\color{magenta}\protect\ensuremath{\protect\bullet},itemsep=\z@}
\setlist[itemize,2]{label=\protect\color{magenta}\protect\ensuremath{\protect\circ},itemsep=\z@}
\setlist[itemize,3]{label=\protect\color{magenta}\protect\ensuremath{\protect\star},itemsep=\z@}
\setlist[enumerate]{leftmargin=*,itemsep=\z@,labelsep=2\p@}
\setlist[enumerate,1]{label=\protect\color{magenta}\Roman*.)}
\setlist[enumerate,2]{itemsep=\p@,label=\protect\color{magenta}\bfseries\arabic*.)}
\setlist[enumerate,3]{itemsep=\p@,label=\protect\color{magenta}\bfseries\alph*.)}
\newenvironment{problem}{\begin{enumerate}}{\end{enumerate}}
\def\pro{\@ifnextchar[{\@opro}{\@pro}}
\def\@opro[#1]{\item{\color{blue}\bfseries(~#1~\label@point~)}}
\def\@pro{\item}
\everymath{\protect\displaystyle\protect\color{blue}}
% ដាក់ពណ៍របស់ Page
\pagecolor{cyan!2!white}
\usepackage{amsmath}
\usepackage{amssymb}
\usepackage{tikz}
\makeatother
\begin{document}
	\begin{multicols}{2}
		ការប្រឡងតេស្ត
		រយៈពេល
	\end{multicols}
	\begin{problem}
		\pro[២៤] គេមានចំនួនកុំផ្លិច $ z_{1}=-3+3\sqrt{3}i $ និង $ z_{2}=2-2\sqrt{3}i $ ។​
			\begin{multicols}{2}
				\begin{problem}
					\pro សរសេរ $ z_{1} $ និង $ z_{2} $​ ជាទម្រង់ត្រីកោណមាត្រ ។
					\pro គណនា $ z_{1} +z_{2} $ និង $ z_{1}-z_{2} $ ។
					\pro គណនា $ z_{1}\times z_{2} $ និង $ \dfrac{z_{1}}{z_{2}} $ ។
				    \pro សរសេរ​ $ z_{1}\times z_{2} $ និង $ \dfrac{z_{1}}{z_{2}} $ ជាទម្រង់ត្រីកោណមាត្រ ។
				\end{problem}
			\end{multicols}
		\pro[២០] គេមានចំនួនកុំផ្លិច $ z_{1}=2-1+\sqrt{3}i $ និង $ z_{2}=1+\cos\dfrac{\pi}{4}-i\sin\dfrac{\pi}{4} $
		\par ចូរសរសេរ $ z_{1} $ និង $ z_{2} $ ជាទម្រង់ត្រីកោណមាត្រ។
		\pro[៣៦] គណនាលីមីតនៃអនុគមន៍ខាងក្រោម៖
			\begin{multicols}{3}
				\begin{problem}
					\pro $ \lim\limits_{x\to 0}\dfrac{\sin20x}{\sin4x} $
					\pro $ \lim\limits_{x\to 0}\dfrac{\tan3x}{\sin10x} $
					\pro $ \lim\limits_{x\to 0}\dfrac{-4x}{\sin2x} $
					\pro $ \lim\limits_{x\to 0}\dfrac{\tan60x}{-2\sin3x} $
					\pro $ \lim\limits_{x\to 0}\dfrac{1-\cos2017x}{x} $
					\pro $ \lim\limits_{x\to 0}\dfrac{1-\cos^2 4x}{x^2} $
					\pro $ \lim\limits_{x\to 0}\dfrac{\sin(\sin(\sin x))}{x} $
					\pro $ \lim\limits_{x\to 0}\dfrac{\cos^2 x -1}{\sin2x} $
					\pro $ \lim\limits_{x\to 0}\dfrac{\sin^2 3x}{1-\cos3x} $
				\end{problem}
			\end{multicols}
		\pro[១៥]គណនាលីមីតនៃអនុគមន៍ខាងក្រោម៖
			\begin{multicols}{2}
				\begin{problem}
					\pro $ \lim\limits_{x\to 0}\dfrac{\sin2x+\tan3x+\sin4x}{\tan5x+\sin6x+\sin7x} $
					\pro $ \lim\limits_{x\to 0}\dfrac{\sin x\cdot\sin2x\cdot\sin3x\cdots\sin20x}{x^{20}} $
					\pro $ \lim\limits_{x\to 0}\dfrac{\sin x +2\sin2x +3\sin3x + \cdots + 20\sin20x}{x} $
					\pro $ \lim\limits_{x\to 0}\dfrac{9x^2-\sin^2 3x}{3x^2-x\sin3x} $
					\pro $ \lim\limits_{x\to 0}\dfrac{x^2+x\sin x}{2x^2+\sin^2 3x} $
				\end{problem}
			\end{multicols}
	\end{problem}
\end{document}