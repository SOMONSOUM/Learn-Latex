\documentclass{beamer}
%
\usetheme[secheader]{KHTUG}
%
\hypersetup{pdfpagemode=FullScreen}
\usepackage{polyglossia}
\newfontfamily{\khmerfont}[%
ItalicFont=*,%
ItalicFeatures={FakeSlant=.12},%
BoldFont=*,%
BoldFeatures={FakeBold=2}]{Khmer.ttf}
\setdefaultlanguage[numerals=khmer]{khmer}
\setsansfont{Khmer.ttf}
\setmonofont{Khmer.ttf}
%
\author[KHTUG]{Khmer \TeX{} Users Group}
\title[beamer]{បង្កើតឯកសារធ្វើបទបង្ហាញ}
%\institute[Short]{Name of the Institute}
\date[21/10/2016]{21~October~2016}
\logo{\includegraphics[width=1cm]{logo.png}}
%
\begin{document}
%
\section{ទំព័រពត៌មាន}
\begin{frame}{ស្វាគមន៏}
	\titlepage
\end{frame}
%
\section{មាតិកា}
\begin{frame}{\insertsection}
	\tableofcontents
\end{frame}
%
\section{បញ្ជី}
\subsection{លេខរៀង}
\begin{frame}{\insertsubsection}
បញ្ជីលេខរៀង
\begin{enumerate}
	\item ចំណុចមួយ
	\item ចំណុចពីរ
\end{enumerate}
%
បញ្ជីអក្សររៀង
\begin{enumerate}[a]
	\item ចំណុចមួយ
	\item ចំណុចពីរ
\end{enumerate}
\end{frame}
\subsection{ចំណុច}
\begin{frame}{\insertsubsection}
បញ្ជីចំណុច
\begin{itemize}
	\item ចំណុចមួយ
	\item ចំណុចពីរ
\end{itemize}
%
បញ្ជីពិពណ៌នា
\begin{description}
	\item[ពាក្យទី១] ចំណុចមួយ
	\item[ពាក្យទី២] ចំណុចពីរ
\end{description}
\end{frame}
%
\section{ប្រអប់}
\subsection{ប្រអប់មានស្រាប់}
\begin{frame}{\insertsubsection}
	\begin{definition}[ឈ្មោះ]
		សរសេរខ្លឹមសារទីនេះ
	\end{definition}
%
	\begin{example}[ឈ្មោះ]
		សរសេរខ្លឹមសារទីនេះ
	\end{example}
\end{frame}
%
\subsection{ប្រអប់មាន ឬគ្មានចំណងជើង}
\begin{frame}{\insertsubsection}
	\begin{block}{ចំណងជើង}
		សរសេរខ្លឹមសារទីនេះ
	\end{block}
%
	\begin{block}{}
		សរសេរខ្លឹមសារទីនេះ
	\end{block}
\end{frame}
%
\section{ចែកជាជួរ}
\begin{frame}{\insertsection}
\begin{columns}
%
	\begin{column}{0.47\textwidth}
		\begin{theorem}
			សរសេរខ្លឹមសារទីនេះ
		\end{theorem}
	\end{column}
%
	\begin{column}{0.47\textwidth}
		\begin{proof}
			សរសេរខ្លឹមសារទីនេះ
            ឋបខឍបឍឃបឍ
	\end{proof}
	\end{column}
\end{columns}
\end{frame}
%
\end{document}