\documentclass[12pt,a4paper]{article}
\usepackage{amsmath}
\usepackage{amssymb}
\begin{document}
	Expand $ A=[(x+1)-y][(x+1)+y] $.
	\par
	Verify that $ 3!+2!<(3+2)! $ and $ 3!-2!>(3-2)! $.
	\par
	The fraction $ 5/10 $ is of ratio $ 1:2 $.
	\par
	If $ f(x)=x^2 $ then show that $ f'(x)=2x $ and $ f''(x)=2 $
	\par $ 16 $ is divisible by $ 2 $ so we write $ 2|16 $.
	\par
	The sum of the inner angles of a triangle is $ 180 $ degree.
	\newline\par
	If the $ \alpha,\beta $ and $ \gamma $ are the inner angles then
	\[ \alpha+\beta+\gamma=\pi \]
	\newline
	The set of positive integer is denoted by $ \mathbb{N} $. \\
	The set of integer is denoted by $ \mathbb{Z} $. \\
	The set of rational, real and complex numbers are denoted by $ \mathbb{Q},\mathbb{R} $ and $ \mathbb{C} $ respectively.\\ [1cm]
	The numbers of permutation of $ k $ elements taken from distinct $ n $
	elements is given by $ P(n,k)=\frac{n!}{(n-k)!} $ \\
	The number of combination of $ k $ elements taken from distinct $ n $
	elements is given by $ C(n,k)=\frac{n!}{k!(n-k)!} $ \\[1cm]
	The relation between combination and permutation is
	\begin{equation}
	C(n,k)=\frac{P(n,k)}{k!}
	\end{equation}
	\begin{enumerate}
		\item Find the formula for computing
		\begin{eqnarray}
			S_{n} &=&1^2+2^2+3^2+\dots+n^n \\
			S_{n}&=&\frac{n(n+1)}{2}
		\end{eqnarray}
		\item The formula is derived from the equality $ (n+1)^{3}-n^{3}=3n^{2}+3n+1 $
		\begin{eqnarray}
		S_{n} &=& 1^{2}+2^{2}+3^{2}+\dots+n^{2}\\
		S_{n} &=& \frac{n(n+1)(2n+1)}{6}
		\end{eqnarray}
		\item To solve quadratic equation $ ax^2+bx+c=0 $ where $ a\ne 0 $ we calculate
		the discriminant
		\begin{center}
				\[ \Delta=b^2-4ac \] or $ \Delta=b^2-4ac $
		\end{center}
	\begin{itemize}
		\item Roots of the equation is given by the formula
		\begin{eqnarray}
		x_{1} &=& \frac{-b-\sqrt{\Delta}}{2a} \\
		x_2 & = & \frac{-b+\sqrt{\Delta}}{2a}
		\end{eqnarray}
		or precisely,
		\begin{equation}
			x=\frac{-b\pm \sqrt{\Delta}}{2a}
		\end{equation}
		Solve $ \sqrt{3}x^2+\sqrt[3]{5}x+\sqrt[4]{7}+\sqrt[5]{5x}=0. $
	\end{itemize}
	\end{enumerate}
\end{document}