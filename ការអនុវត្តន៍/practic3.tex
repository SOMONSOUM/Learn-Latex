\documentclass[a5paper, 10pt]{book}
\usepackage{amsmath}
\usepackage{xcolor}
\usepackage{amssymb}
\begin{document}
	There are two way of entering new paragraph. One way is entering new one
		or more blank lines in code. 
		\begin{flushleft}
			The paragraph is justified by default. So we do not need to use any
			command for this purpose. For instance, this short paragraph is
			justified. To check it, see the left and right margin.
		\end{flushleft}
		\begin{flushright}
	 	This text is right aligned. What happen to this sample text? This text
		is right aligned. What happen to this sample text? This text is
		right aligned. What happen to this sample text? This text is right
		aligned. What happen to this sample text? This text is right aligned
		. What happen to this sample text? This text is right aligned. What
		happen to this sample text?
		\end{flushright}
		\newpage
	Another way is using command \par like this example. Notice that there is
		an automatically in indent in every new paragraph.
		\begin{center}
			 This text is centered. What happen to this sample text? This text is
			centered. What happen to this sample text? This text is centered.
			What happen to this sample text? This text is centered. What happen
			to this sample text? This text is centered. What happen to this
			sample text? This text is centered. What happen to this sample text?
		\end{center}
\end{document}