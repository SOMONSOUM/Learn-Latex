%\title{Hai câu trắc nghiệm mẫu}
\documentclass[12pt]{examdesign}
\usepackage{fourier}
\usepackage{amsmath,amsxtra,latexsym, amssymb, amscd}
\usepackage[mathletters]{ucs}
\usepackage[utf8x]{inputenc}
\usepackage[utf8]{vietnam}
\usepackage{color}
\usepackage{graphicx}
\usepackage{wrapfig}
\usepackage{times}
\usepackage{dethi} 
\usepackage[a4paper,tmargin=1.0cm, bmargin=1.5cm, lmargin=1.5cm, rmargin=1.5cm]{geometry}
\ContinuousNumbering 
\ShortKey
\NumberOfVersions{1} % số bài thi khác nhau được in ra
\SectionPrefix{\relax }
% Tiêu đề 
\tentruong{SỞ GIÁO DỤC VÀ ĐÀO TẠO BÌNH DƯƠNG}
\loaidethi{ĐỀ THI MẪU} % ĐỀ THI CHÍNH THỨC 
\Sotrang{Đề thi gồm có 5 trang}  % nếu sửa ở đây thì sửa luôn ở dòng 35
\tenkythi{KỲ THI THỬ THPT QUỐC GIA NĂM 2017}
\tenmonhoc{Môn: Toán}
\madethi{100}\bigskip

\thoigian{\underline{Thời gian làm bài: 90 phút, không kể thời gian phát đề}}

% hết tiêu đề 

\tieudetracnghiem
\tieudedapan
\tieudeduoi
\daungoac{}{.}
\chucauhoi{Câu} 
\mauchu{black}
\sotrang{5}
\renewcommand{\baselinestretch}{1.125}
\NoRearrange

\newcommand{\chenhinhve}[3]{
\begin{wrapfigure}{r}{#1}\examvspace*{#3}\includegraphics[width=#1]{#2}\end{wrapfigure}
}



\begin{document}

\begin{vnmultiplechoice}[ rearrange=yes, keycolumns=5]%

\begin{question}%1
Cho hình chóp S.ABCD có đáy ABCD là hình vuông cạnh a, mặt bên SAB là tam giác đều và nằm trong mặt phẳng vuông góc với đáy. Thể tích của khối chóp S.ABCD là:
 
\datcot
\bonpa
{\sai{$\dfrac{a^3\sqrt3}{2}$.}}
{\sai{$\frac{1}{3}a^3$.}}
{\dung{$\dfrac{a^3\sqrt3}{6}$.}}
{\sai {$\dfrac{a^3\sqrt3}{3}$.}}
\end{question}

\begin{question}%2
Cho hình chóp SABCD có đáy ABCD là hình vuông có $AC = a\sqrt 2$. SA vuông góc với mặt đáy và mặt bên SCD hợp với đáy một góc $60^0$. Thể tích của khối chóp S.ABCD là:
 
\datcot
\bonpa
{\dung{$\dfrac{a^3\sqrt3}{3}$.}}
{\sai{$a^3 \sqrt 3$.}}
{\sai{$\dfrac{a^3\sqrt6}{3}$.}}
{\sai {$\dfrac{2a^3\sqrt6}{3}$.}}
\end{question}

\begin{question}%3
Cho hình chóp đều SABCD có độ dài cạnh đáy bằng $a\sqrt 3$ và cạnh bên tạo với đáy một góc $60^0$. Thể tích của khối chóp S.ABCD là:
 
\datcot
\bonpa
{\sai{$3a^3 \sqrt 2$.}}
{\dung{$\dfrac{3a^3\sqrt2}{2}$.}}
{\sai{$\dfrac{a^3\sqrt2}{2}$.}}
{\sai {$\dfrac{a^3\sqrt6}{2}$.}}
\end{question}

\begin{question}%4
Cho hình chóp đều SABCD có độ dài cạnh đáy bằng a và mặt bên tạo với đáy một góc $60^0$. Thể tích của khối chóp S.ABCD là:
 
\datcot
\bonpa
{\sai{$\dfrac{a^3\sqrt6}{6}$.}}
{\sai{$\dfrac{a^3\sqrt3}{2}$.}}
{\sai {$\dfrac{a^3\sqrt6}{3}$.}}
{\dung{$\dfrac{a^3\sqrt3}{6}$.}}
\end{question}

\begin{question}%5
Cho hình chóp SABC có đáy ABC là tam giác cân tại A, $BC = 2a\sqrt 3$ và góc $\widehat{BAC} = 120^0$. SA vuông góc với đáy và SA = a. Thể tích của khối chóp S.ABC là:  
 
\datcot
\bonpa
{\sai{$a^3 \sqrt 3$.}}
{\dung{$\dfrac{a^3\sqrt3}{3}$.}}
{\sai{$\dfrac{2a^3\sqrt3}{3}$.}}
{\sai {$\dfrac{a^3\sqrt3}{6}$.}}
\end{question}

\begin{question}%6
Cho hàm số $y = \frac{{x^2  - x - 2}}{{x + 2}}$. Tiếp tuyến của đồ thị hàm số song song với đường thẳng 3x + y - 2 = 0 là    
 
\datcot[2]
\bonpa
{\dung{$y=-3x-3; y=-3x-19$.}}
{\sai{$y=-3x+5$.}}
{\sai{$y=-3x-3$.}}
{\sai {$y=-3x+5;y=-3x-3$.}}
\end{question}

\begin{question}%7
Cho hàm số $y = x^3-3x^2+2$ có đồ thị là (C). Tiếp tuyến của (C) tại giao điểm của (C) với trục Oy có phương trình:    
 
\datcot
\bonpa
{\dung{$y=2$.}}
{\sai{$y=0$.}}
{\sai{$x+y=0$.}}
{\sai {$x-2y=0$.}}
\end{question}

\begin{question}%8
Cho hàm số $y = \frac{{2x + 3}}{{2x - 1}}$ có đồ thị là (C). Số tiếp tuyến của (C) vuông góc với đường thẳng $y = \frac{1}{2}x$ là:    
 
\datcot
\bonpa
{\dung{$2$.}}
{\sai{$1$.}}
{\sai{$0$.}}
{\sai {$3$.}}
\end{question}

\begin{question}%9
Cho hàm số $y = \frac{{4}}{{x - 1}}$ có đồ thị là (C). Phương trình tiếp tuyến của (C) tại điểm có hoành độ x = -1 là:    
 
\datcot
\bonpa
{\dung{$y=-x-3$.}}
{\sai{$y=-x+2$.}}
{\sai{$y=x-1$.}}
{\sai {$y=x+2$.}}
\end{question}


\begin{question}%10
Cho hàm số $y = \frac{{1}}{{3}}x^3+3x^2-2$ có đồ thị là (C). Phương trình tiếp tuyến của (C) có hệ số góc k = -9 là:    
 
\datcot
\bonpa
{\dung{$y=-9x-43$.}}
{\sai{$y=-9x+43$.}}
{\sai{$y=-9x-11$.}}
{\sai {$y=-9x-27$.}}
\end{question}

\end{vnmultiplechoice}
\end{document}







\end{vnmultiplechoice}
\end{document}


 



