\documentclass[12pt,a4paper]{article}
\usepackage[margin=1.25cm]{geometry}
\usepackage{mathpazo}% change math font
\usepackage{enumitem}% change list environment like enumerate, itemize and description
\usepackage{multicol}% multi columns
\usepackage{tikz}% graphic drawing
\usepackage[no-math]{fontspec}% font specfication
\setmainfont{Khmer OS Content}% set default font to Khmer OS Content
%
\SetEnumitemKey{I}{%
leftmargin=*,
label={\protect\tikz[baseline=-0.9ex]\protect\node[draw=gray,thick,circle,minimum height=.65cm,inner sep=1pt,text=black,fill=white]{\Roman*};},%
font=\small\sffamily\bfseries,%
labelsep=1ex,%
topsep=0pt}
%
\SetEnumitemKey{a}{%
leftmargin=*,%
label={\protect\tikz[baseline=-0.9ex]\protect\node[draw=gray,thick,circle,minimum height=.5cm,inner sep=1pt,text=black,fill=white]{\alph*};},%
font=\small\sffamily\bfseries,%
labelsep=1ex,%
topsep=0pt}
%
\SetEnumitemKey{1}{leftmargin=*,%
label={\protect\tikz[baseline=-0.9ex]\protect\node[draw=gray,thick,circle,minimum height=.5cm,inner sep=1pt,text=black,fill=white]{\arabic*};},%
font=\small\sffamily\bfseries,%
labelsep=1ex,%
topsep=0pt}
%
\def\hard{\leavevmode\makebox[0pt][r]{\large\ensuremath{\star}\hspace{2em}}}
%
\def\hhard{\leavevmode\makebox[0pt][r]{\large\ensuremath{\star\star}\hspace{2em}}}
%
\everymath{\protect\displaystyle\protect\color{blue}}
% ដាក់ពណ៍របស់ Page
\pagecolor{cyan!2!white}
%
\begin{document}
\begin{enumerate}[I]
\item ($ 15 $ ពិន្ទុ) គេមានចំនួនកុំផ្លិច $ z=(\sqrt{6}+\sqrt{2})+i(\sqrt{6}-\sqrt{2}) $~។
\begin{enumerate}[1]
\item សរសេរ $ z^2 $ ជាទម្រង់ត្រីកោណមាត្ររួចទាញរកទម្រង់ត្រីកោណមាត្រនៃ $ z $~។
\item រកចំនួនគត់ $ n $ វិជ្ជមានតូចបំផុតដែល $ z^n $ ជាចំនួនពិត។
\end{enumerate}
\item ($ 15 $ ពិន្ទុ) គណនាលីមីតខាងក្រោម៖
\begin{multicols}{3}
\begin{enumerate}[a]
\item $ \lim\limits_{x\to 3}\frac{x^2-4x+3}{9-x^2} $
\item $ \lim\limits_{x\to 2}\frac{x\sqrt{x}-2\sqrt{2}}{\sqrt{x}-\sqrt{2}} $
\item $ \lim\limits_{x\to 0}\frac{e^{x^2}+\sin (x^2)-1}{2x\sin x} $
\end{enumerate}
\end{multicols}
\item ($ 15 $ ពិន្ទុ) គេមានអនុគមន៍ $ f(x)=\frac{x^2+3x-7}{(x+2)(x-1)^2} $~។
\begin{enumerate}[1]
\item កំណត់ចំនួនពិត $ a,b,c $ ដែល $ f(x)=\frac{a}{x-1}+\frac{b}{(x-1)^2}+\frac{c}{x+2} $
\item គណនាអាំងតេក្រាល $ \int f(x)\mathrm{d}x $~។
\end{enumerate}
\item ($ 15 $ ពិន្ទុ) នៅក្នុងកន្ត្រកមួយមានពងទា``កូន''ចំនួន $ 5 $ គ្រាប់ ពងទា``សាប''ចំនួន $ 7 $ គ្រាប់ និងពងទា``ខូច''ចំនួន $ 3 $ គ្រាប់។ ក្មេងម្នាក់ចាប់យកពងទា $ 5 $ គ្រាប់ ដោយចៃដន្យពីក្នុងកន្ត្រកនោះ។\\
គណនាប្រូបាបនៃព្រឹត្តិការណ៍៖
\begin{enumerate}[a]
\item $ A: $ ``បានពងទាកូន $ 2 $ គ្រាប់ ពងទាសាប $ 2 $ គ្រាប់ និងពងទាខូច $ 1 $ គ្រាប់''
\item $ B: $ ``បានពងទាកូន $ 4 $ គ្រាប់''
\item $ C: $ ``បានពងទាខូចយ៉ាងតិច $ 1 $ គ្រាប់''
\end{enumerate}
\item\hhard ($ 35 $ ពិន្ទុ) 
\begin{itemize}
\item[ផ្នែក A.)] គេមានអនុគមន៍ $ g $ កំណត់លើ $ (0,+\infty) $ ដោយ $ g(x)=x^2+1-\ln x $~។
\begin{enumerate}[a]
\item គណនាដេរីវេ $ g'(x) $ នៃអនុគមន៍ $ g(x) $ រួចទាញរកអថេរភាពនៃ $ g $~។
\item គូសតារាងអថេរភាពនៃ $ g $ ហើយទាញរកសញ្ញានៃ $ g $~។
\end{enumerate}
\item[ផ្នែក B.)] គេមាន $ f $ ជាអនុគមន៍កំណត់លើ $ (0,+\infty) $ ដោយ $ f(x)=1-x-\frac{\ln x}{x} $ ហើយមានក្រាប $ C $~។
\begin{enumerate}[a]
\item គណនា $ f'(x) $ ហើយទាញ $ f'(x) $ ជាអនុគមន៍នៃ $ g(x) $ ព្រមទាំងបញ្ជាក់សញ្ញា $ f'(x) $ លើ $ (0,+\infty) $~។
\item គណនាលីមីត​នៃអនុគម៍ $ f $ ត្រង់ $ 0^+ $ និង $ +\infty $ រួចគូសតារាងអថេរភាពនៃ $ f $~។
\item បង្ហាញថាបន្ទាត់ $ d:\;y=-x+1 $ ជាអាស៊ីមតូតទ្រេតនៃក្រាប $ C $ ខាងមែក $ +\infty $~។\\
រួចសិក្សាទីតាំងរវាងក្រាប $ C $ និងបន្ទាត់ $ d $~។
\item គូសខ្សែកោង $ C $ និងបន្ទាត់ $ d $ ក្នុងតម្រុយតែមួយ។
\end{enumerate}
\end{itemize}
\item\hard ($ 30 $ ពិន្ទុ) 
\begin{itemize}
\item[ផ្នែក A.)] គេឲ្យសមីការទូទៅនៃអេលីប $ E:\;9x^2+25y^2=225 $~។
\begin{enumerate}[a]
\item រកប្រវែងអ័ក្សធំ ប្រវែងអ័ក្សតូច និងកូអរដោនេកំពូលទាំងពីរ។
\item សង់អេលីប​ $ E $~។
\end{enumerate}
\item[ផ្នែក B.)] ក្នុងលំហរប្រដាប់ដោយតម្រុយអតូណរម៉ាល់មានទិសដៅវិជ្ជមាន $ (O,\vec{i},\vec{j},\vec{k}) $ គេមានចំណុច​\\
$ A(1,0,1),B(2,1,2),C(2,3,1) $ និង $ D(1,2,3) $~។
\begin{enumerate}[a]
\item សរសេរវ៉ិចទ័រ $ \overrightarrow{AB},\overrightarrow{AC},\overrightarrow{AD} $ រួចគណនា $ \overrightarrow{AB}\times\overrightarrow{AC} $ និង $ (\overrightarrow{AB}\times\overrightarrow{AC})\cdot\overrightarrow{AD} $
\item សរសេរសមីការទូទៅនៃប្លង់ $ ABC $ ហើយបង្ហាញថា $ D $ មិនមែនជាចំណុចនៃប្លង់ $ ABC $
\item សរសេរសមីការឆ្លុះនៃបន្ទាត់ $ L $ កាត់តាម $ D $ ហើយកែងនឹងប្លង់ $ ABC $~។
\end{enumerate}
\end{itemize}
\end{enumerate}
\end{document}